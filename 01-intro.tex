\chapter*{ВВЕДЕНИЕ}
\addcontentsline{toc}{chapter}{ВВЕДЕНИЕ}

В наше время аппаратура для регистрации изображений повсеместно распространена. Практически на каждом шагу можно встретить смартфоны с фронтальными и тыловыми камерами, видеокамеры в автомобилях, на улицах, общественном транспорте, жилых домах, магазинах и т.д \cite{cameras}.

Изображения в процессе формирования их системами обычно подвергаются воздействию различных случайных помех или шумов. Фундаментальной проблемой в области обработки изображений является эффективное удаление шума при сохранении важных для последующего распознавания деталей изображения.

Цель данной научно-исследовательской работы --- провести классификацию существующих алгоритмов фильтрации помех на изображениях.

Для достижения указанной выше цели следует выполнить задачи:
\begin{itemize}
	\item изучить общие понятия о шумовых помехах на изображениях;
	\item рассмотреть различные виды шумов;
	\item изучить существующие методы фильтрации шумов;
	\item сформулировать критерии для сравнения рассмотренных алгоритмов;
	\item классифицировать алгоритмы устранения помех.
\end{itemize}