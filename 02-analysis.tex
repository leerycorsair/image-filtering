\chapter{Анализ предметной области}


\section{Общие сведения о построении изображений}

Основным устройством для получения изображений является камера, среди которых выделяются две категории: пленочные и цифровые. 

В пленочных камерах для получения изображений используется пленка. Пленка представляет собой полосу светочувствительного материала, который подвергается воздействию света при съемке. Когда пленка проявляется, экспонированные участки преобразуются в видимое изображение. Пленочные фотоаппараты распространены меньше, чем цифровые, но некоторые фотографы все еще предпочитают их за уникальный внешний вид и творческие возможности \cite{filmcameras}.

Цифровые камеры используют датчик изображения для улавливания света и преобразования его в цифровой вид. Датчик изображения представляет собой микросхему, содержащую миллионы крошечных светочувствительных транзисторов, называемых фотодиодами. Когда свет попадает на фотодиод, он генерирует электрический заряд, который пропорционален интенсивности света.

Датчик изображения представляет собой сетку пикселей, каждый из которых соответствует одному фотодиоду. Интенсивность света, падающего на каждый пиксель, записывается в виде цифрового значения, называемого значением пикселя. Совокупность значений пикселей для всех пикселей датчика изображения называется цифровым изображением.

Цифровые камеры включают в себя объектив, который фокусирует свет от сцены на датчик изображения. Диафрагма объектива описывает количество света, попадающего в камеру, а затвор --- время, в течение которого датчик изображения подвергается воздействию света.

Помимо датчика изображения и объектива, цифровые фотоаппараты оснащены электроникой, которая управляет такими функциями фотоаппарата, как фокусировка, баланс белого и обработка изображения \cite{digitalcameras}.

Цифровые камеры обладают рядом преимуществ по сравнению с пленочными \cite{fvsdcameras}:

\begin{itemize}
    \item \textit{Удобство}. Цифровые камеры позволяют просматривать и редактировать снимки сразу после съемки, т.е. не нужно ждать, пока пленка будет проявлена, чтобы увидеть результаты съемки.
    \item \textit{Хранение и совместное использование}. Цифровые камеры позволяют хранить снимки на карте памяти или компьютере и легко делиться ими по электронной почте, в социальных сетях или на других цифровых платформах. При использовании пленочных фотоаппаратов необходимо проявить и распечатать пленку, прежде чем можно будет просмотреть или поделиться снимками.
    \item \textit{Стоимость}. Использование цифровых камер обычно обходится дешевле, чем пленочных, поскольку отсутствует необходимость покупать пленку или платить за ее обработку.
    \item \textit{Универсальность}. Цифровые камеры позволяют настраивать экспозицию, баланс белого и другие параметры изображения после того, как оно было снято. В пленочных фотоаппаратах необходимо правильно настроить экспозицию и другие параметры во время съемки, поскольку после этого их нельзя отрегулировать.
    \item \textit{Качество}. Современные цифровые камеры способны создавать высококачественные изображения с широким динамическим диапазоном и точными цветами.
    \item \textit{Скорость}. Цифровые камеры, как правило, быстрее пленочных, поскольку они могут делать несколько снимков подряд. Это полезно при съемке быстро движущихся объектов или для того, чтобы сделать несколько снимков для увеличения шансов получить хорошее изображение.
\end{itemize}

Помехами на изображениях называют случайное изменение яркости или цвета произвольного пикселя, которое может быть вызвано несколькими факторами. Шум изображения можно увидеть в виде зернистого или крапчатого рисунка на изображении, и он может быть особенно заметен в областях с однородным цветом или слабым освещением \cite{basicnoise}.

С физической точки зрения шумы на изображениях могут возникать по следующим причинам\cite{noisephysics}:

\begin{itemize}
    \item \textit{Тепловой шум}. Тепловой шум вызван случайным движением электронов внутри датчика изображения. При повышении температуры датчика изображения движение электронов усиливается, что приводит к увеличению теплового шума.
    \item \textit{Квантовый шум}. Квантовый шум вызван случайностью квантово-механических процессов, происходящих в датчике изображения. Этот шум присущ датчику изображения и присутствует даже при низком уровне освещенности.
    \item \textit{Шум считывания}. Шум считывания вызван электрическим шумом, вносимым схемой, которая считывает данные изображения с датчика изображения. Этот шум чаще возникает при высоких значениях светочувствительности, так как схема вынуждена усиливать сигнал с датчика изображения.
\end{itemize}

\section{Виды помех на изображениях}

Существует несколько типов шумов, которые могут возникать на изображении:

\begin{enumerate}
    \item \textit{\textbf{Гауссовский шум}}. Гауссовский шум --- тип шума, который часто встречается в системах обработки сигналов и связи. Это тип статистического шума, который характеризуется функцией плотности вероятности, которая следует гауссовскому распределению, также известному как нормальное распределение. Это означает, что шум описывается колоколообразной кривой, при этом наиболее вероятные значения находятся вблизи среднего значения, а вероятность отклонения значений от среднего значения уменьшается по мере увеличения расстояния от среднего значения \cite{noisetypes}.

    В математических терминах гауссовский шум может быть описан следующей функцией плотности вероятности:

    \begin{equation}
        f(x) = \frac{1}{\sqrt{2 \pi \sigma^2}} \exp \left( -\frac{(x - \mu)^2}{2 \sigma^2} \right),
    \end{equation}
    где
    \begin{itemize}
        \item $p(x)$ --- функция плотности вероятности шума;
        \item $x$ --- значение шума;
        \item $\mu$ --- среднее значение шума, также известное как ожидаемое значение или среднее значение;
        \item $\sigma^2$ --- дисперсия шума, которая определяет разброс распределения шума вокруг среднего значения.
    \end{itemize}
    
    Гауссов шум часто встречается в электронных системах, например, в цифровых изображениях или аудио сигналах, где он может быть вызван различными источниками, такими как тепловой шум и электронные помехи. Он также часто используется в качестве модели статистического шума в различных приложениях, таких как обработка изображений и системы связи.

    На рисунках \ref{fig: s11}--\ref{fig: s12} представлен пример шума Гаусса.

    \putimage{sydney_original}{h}{0.65}{Исходное изображение}{s11}
    \putimage{sydney_n_gaussian}{h}{0.65}{Гауссов шум}{s12}

    \item \textit{\textbf{Солевой и перечный шум}}. Этот тип шума проявляется в виде беспорядочно разбросанных черных и белых пикселей на изображении, т.е. характеризуется наличием изолированных пикселей с экстремальными значениями интенсивности. Эти пиксели могут значительно исказить внешний вид изображения и нарушить его визуальное качество \cite{noisetypes}.

    Данный шум может быть описан следующей функцией плотности вероятности:

    \begin{equation}
        p(x) = p_{salt} \delta(x - v_{salt}) + p_{pepper} \delta(x - v_{pepper}),
    \end{equation}
    где
    \begin{itemize}
        \item $p(x)$ --- функция плотности вероятности шума;
        \item $x$ --- значение шума;
        \item $p_{salt}$ и $p_{pepper}$ --- вероятности пикселей <<соль>> и <<перец>>, соответственно;
        \item $\delta(x)$ --- дельта-функция Дирака, математическая функция, которая равна 0 везде, кроме точки 0, где она бесконечна;
        \item  $v_{salt}$ и $v_{pepper}$ --- значения интенсивности <<солевого>> и <<перечного>> пикселей, соответственно.
    \end{itemize}
    
    Часто встречается на изображениях, полученных низкокачественными датчиками, например, в камерах наблюдения или в условиях недостаточной освещенности. Он также может быть вызван другими факторами, например, электронными помехами или повреждением данных. Следует отметить, что данный шум трудно удалить с изображений, поскольку он, как правило, затрагивает небольшое количество пикселей на изображении, а традиционные методы сглаживания могут размыть важные детали на изображении.

    На рисунках \ref{fig: s21}--\ref{fig: s22} представлен пример солевого и перечного шума.

    \putimage{sydney_original}{h}{0.65}{Исходное изображение}{s21}
    \putimage{sydney_n_sp}{h}{0.65}{Солевой и перечный шум}{s22}

    \item \textit{\textbf{Спекл-шум}}. Спекл-шум, также известный как мультипликативный шум или когерентный шум, --- это тип шума, который часто встречается на изображениях, полученных с помощью систем когерентной визуализации. Он характеризуется наличием небольших, беспорядочно распределенных светлых или темных пятен на изображении, часто называемых <<крапинками>> \cite{noisetypes}.
    
    Данный вид шума присутствует на ультразвуковых изображениях, где он вызван интерференцией ультразвуковых волн, отраженных от различных структур в теле. Спекл-шум бывает трудно удалить с изображений, поскольку он затрагивает большое количество пикселей на изображении, а традиционные методы сглаживания могут размыть важные детали на изображении.

    На рисунках \ref{fig: s31}--\ref{fig: s32} представлен пример спекл-шума.

    \putimage{sydney_original}{h}{0.65}{Исходное изображение}{s31}
    \putimage{sydney_n_speckle}{h}{0.65}{Спекл-шум}{s32}
\end{enumerate}

Эти типы шума могут иметь четкие характеристики, что облегчает их выявление и устранение. Однако часто бывает так, что реальные цветные фотографии содержат смесь различных типов шума, что усложняет определение конкретной причины дефектов на изображении. Для точного обнаружения и устранения дефектных пикселей важно учитывать различные факторы, которые могут способствовать появлению шума на изображении, такие как устройство, использованное для захвата изображения, и этапы обработки, задействованные при создании изображения. Понимание специфических характеристик шума, присутствующего на изображении, может помочь более эффективно удалить дефектные пиксели и улучшить общее качество изображения.

\chapter{Классификация методов фильтрации шумов}

Методы фильтрации шумов на изображениях можно разделить на два основных класса \cite{filtertypes}:

\begin{enumerate}
    \item \textbf{\textit{Линейные фильтры изображений}} --- это методы обработки изображений, которые применяют линейное преобразование к значениям пикселей в изображении. Это означает, что выходные значения пикселей представляют собой линейную комбинацию входных значений пикселей с некоторыми коэффициентами (весами), применяемыми к каждому пикселю. Линейные фильтры могут использоваться для выполнения широкого спектра задач по обработке изображений, таких как сглаживание, повышение резкости, определение краев и подавление шума.

    \item \textbf{\textit{Нелинейные фильтры изображений}} --- это методы обработки изображений, которые не применяют линейное преобразование к значениям пикселей. Эти фильтры могут быть более сложными и могут включать применение различных весов к различным пикселям на основе интенсивности или цвета пикселей, или применение различных преобразований к различным областям изображения. Нелинейные фильтры могут использоваться для выполнения широкого спектра задач обработки изображений, таких как обнаружение краев, усиление контраста и цветокоррекция.
\end{enumerate}

\section{Линейные методы}
\subsection{Метод среднего арифметического}

Фильтр среднего арифметического --- это простой метод фильтрации, который обычно используется для сглаживания изображений и уменьшения влияния шума. Он работает путем замены значения каждого пикселя изображения средним значением окружающих его пикселей, взвешенных в соответствии с ядром \cite{meanfilter}.

Математически фильтр среднего арифметического можно представить в виде следующей формулы:
\begin{equation}
    I_{filtered}(x,y) = \frac{1}{K} \sum_{i=-\frac{N}{2}}^{\frac{N}{2}} \sum_{j=-\frac{N}{2}}^{\frac{N}{2}} K(i,j)\cdot I(x+i,y+j),
\end{equation}
где
\begin{itemize}
    \item $I_{filtered}(x,y)$ --- изображение, после применения фильтрации;
    \item $I(x,y)$ --- исходное изображение;
    \item $K(i,j)$ --- ядро, которое представляет собой матрицу весов, определяющую, какой вклад вносит каждый окружающий пиксель в среднее значение;
    \item $N$ --- размер ядра, который обычно является нечетным числом.
\end{itemize}

Фильтр среднего арифметического --- примитивный и неэффективный метод сглаживания изображений и уменьшения шума, т.к. он также может размыть важные детали изображения.

\subsection{Метод фильтрации по Гауссу}
Фильтр Гаусса --- это линейный фильтр, который работает путем свертки изображения с ядром, которое представляет собой небольшую матрицу весов. Веса в ядре определяются функцией Гаусса \cite{gaussianfilter}.

Фильтр Гаусса описывается следующим уравнением:
\begin{equation}
    G(x,y) = \frac{1}{2\pi\sigma^2}e^{-\frac{x^2+y^2}{2\sigma^2}},
\end{equation}
где 
\begin{itemize}
    \item $G(x,y)$ --- значение гауссовой функции в позиции $(x,y)$;
    \item $\sigma$ --- стандартное отклонение.
\end{itemize}

Чтобы применить гауссовский фильтр к изображению, выполняется свертка изображения с ядром в каждом из пикселей:

\begin{equation}
I_{i,j}' = \sum_{m=-\infty}^{\infty} \sum_{n=-\infty}^{\infty} I_{i+m,j+n} G_{m,n},
\end{equation}
где 
\begin{itemize}
    \item $I_{i,j}'$ --- значение пикселя в позиции $(i,j)$ после применения гауссова фильтра;
    \item $I_{i+m,j+n}$ --- значение пикселя в позиции $(i+m,j+n)$ в исходном изображении;
    \item $G_{m,n}$ --- значение ядра в позиции $(m,n)$.
\end{itemize}

Гауссовский фильтр обладает рядом свойств, которые делают его полезным для обработки изображений. Одно из этих свойств заключается в том, что он является низкочастотным фильтром, что означает, что его можно использовать для удаления высокочастотных компонентов из изображения, сохраняя при этом низкочастотные компоненты. Еще одним свойством фильтра Гаусса является его разделимость, т.е. он может быть реализован более эффективно путем применения фильтра отдельно в горизонтальном и вертикальном направлениях.

Одно из ограничений гауссовского фильтра заключается в том, что он плохо подходит для сохранения резких краев изображения.

\section{Нелинейные методы}

\subsection{Метод медианной фильтрации}

Медианный фильтр --- это метод фильтрации, который работает путем замены значения каждого пикселя изображения медианным значением окружающих его пикселей в соответствии с ядром \cite{medianfilter}. 

Медианное значение --- это среднее значение в наборе значений, такое, что половина значений меньше, а половина значений больше.

Математически медианный фильтр можно представить в виде следующей формулы:

\begin{equation}
    I_{filtered}(x,y) = m   \begin{pmatrix}
I(x-1,y-1) && I(x,y-1) && I(x+1,y-1) \\
I(x-1,y) && I(x,y) && I(x+1,y) \\
I(x-1,y+1) && I(x,y+1) && I(x+1,y+1)
\end{pmatrix},
\end{equation}
где
\begin{itemize}
    \item $I_{filtered}(x,y)$ --- фильтрованное изображение;
    \item $I(x,y)$ --- исходное изображение.
\end{itemize}

Эта формула представляет медианный фильтр 3x3, что является обычным размером ядра медианного фильтра. Медианный фильтр может быть расширен до больших размеров ядра путем добавления большего количества окружающих пикселей к набору значений. Следует отметить, что фильтрационное окно может быть произвольной геометрической формы.

\subsection{Метод билатеральной фильтрации}

Билатеральный фильтр --- это нелинейный фильтр, который используется для сглаживания изображения с сохранением краев и является модификацией фильтра Гаусса. Фильтрация выполняется путем замены значения каждого пикселя средневзвешенным значением значений близлежащих пикселей, где вес пикселя определяется комбинацией его интенсивности и пространственного расстояния от центрального пикселя \cite{bilateralfilter}.

Билатеральный фильтр определяется следующим образом:

\begin{equation}
	I_\text{filtered}(x) = \frac{1}{W_p} \sum_{x_i \in \Omega} I(x_i)f_r(\|I(x_i) - I(x)\|)g_s(\|x_i - x\|),
\end{equation}

\begin{equation}
	W_p = \sum_{x_i \in \Omega}{f_r(\|I(x_i) - I(x)\|)g_s(\|x_i - x\|)},
\end{equation}
где
\begin{itemize}
	\item $I_\text{filtered}$ --- изображение, после применения фильтрации;
	\item $I$ --- исходное изображение;
	\item $x$ --- координаты текущего пикселя для фильтрации;
	\item $\Omega$ --- окно с центром в $x$, тогда $x_i \in \Omega$ соседний пиксель;
	\item $f_r$ --- интенсивности пикселей;
	\item $g_s$ --- функция Гаусса.
\end{itemize}

Пиксель просто заменяется взвешенным средним его соседей.

\noindent
\begin{equation}
	w(i, j, k, l) = \exp\left(-\frac{(i - k)^2 + (j - l)^2}{2 \sigma_d^2} - \frac{\|I(i, j) - I(k, l)\|^2}{2 \sigma_r^2}\right),
\end{equation}
где $\sigma_d$ и $\sigma_r$ --- сглаживающие параметры, и $I(i, j)$ и $I(k, l)$ --- интенсивности пикселей $(i, j)$ и $ (k, l)$ соответственно.

После вычисления весов, необходимо нормализовать их:
\begin{equation}
	I_D(i, j) = \frac{\sum_{k, l} I(k, l) w(i, j, k, l)}{\sum_{k, l} w(i, j, k, l)},
\end{equation}
где $I_D$ интенсивность пикселя $(i, j)$ без шума.

\section{Сравнение методов}

\textbf{\textit{Пиковое отношение сигнал/шум (PSNR)}}--- это мера качества восстановленного изображения по сравнению с исходным\cite{metrics}. PSNR обычно используется для оценки эффективности алгоритмов обработки изображений и может быть рассчитан по следующей формуле:

\begin{equation}
    PSNR = 10 \log_{10} \frac{MAX_{I}^2}{MSE},
\end{equation}
где
\begin{itemize}
    \item $PSNR$ --- пиковое отношение сигнал/шум;
    \item $MAX_{I}$ --- максимально возможное значение пикселя изображения;
    \item $MSE$ --- средняя квадратичная ошибка между исходным и восстановленным изображением.
\end{itemize}

Средняя квадратичная ошибка определяется как:
\begin{equation}
    MSE = \frac{1}{mn} \sum_{i=1}^{m} \sum_{j=1}^{n} (I_{i,j} - I'_{i,j})^2,
\end{equation}
где
\begin{itemize}
    \item $m$ и $n$ --- размеры изображения;
    \item $I_{i,j}$ --- значение пикселя исходного изображения в позиции $(i,j)$;
    \item $I'_{i,j}$ --- значение пикселя восстановленного изображения в позиции $(i,j)$.
\end{itemize}

PSNR обычно выражается в децибелах (дБ) и рассчитывается по логарифмической шкале. Более высокое значение PSNR указывает на более высокое качество реконструкции, при этом максимально возможное значение составляет $PSNR = \infty$ для идеальной реконструкции. На практике значение PSNR 30 дБ или выше обычно считается хорошим качеством, в то время как значение ниже 20 дБ обычно считается плохим качеством.

PSNR --- полезная метрика для сравнения производительности различных алгоритмов обработки изображений, но у нее есть некоторые ограничения. Одно из этих ограничений заключается в том, что он не учитывает человеческое восприятие качества изображения, на которое могут влиять такие факторы, как пространственное распределение ошибок и наличие визуально заметных особенностей. В результате PSNR не всегда точно отражает субъективное качество изображения.

\textbf{\textit{Метрика структурного сходства изображений (SSIM)}} основана на идее  о том, что зрительная система человека обладает высокой адаптивностью и может мириться с некоторым ухудшением качества изображения при условии, что ухудшение не слишком сильное и общая структура изображения сохраняется \cite{metrics}. Для количественной оценки этой идеи индекс SSIM сравнивает яркость, контрастность и структуру двух изображений и объединяет эти сравнения в один балл.

Индекс SSIM определяется следующим образом:
\begin{equation}
    SSIM(x, y) = \frac{(2\mu_x\mu_y + c_1)(2\sigma_{xy} + c_2)}{(\mu_x^2 + \mu_y^2 + c_1)(\sigma_x^2 + \sigma_y^2 + c_2)},
\end{equation}
где
\begin{itemize}
    \item $x$ и $y$ --- два сравниваемых изображения;
    \item $\mu_x$ и $\mu_y$ --- средние значения пикселей в $x$ и $y$;
    \item $\sigma_x^2$ и $\sigma_y^2$ --- дисперсии значений пикселей в $x$ и $y$;
    \item $\sigma_{xy}$ --- ковариация значений пикселей в $x$ и $y$;
    \item $c_1$ и $c_2$ --- константы, используются для стабилизации разделения и обычно устанавливаются равными $c_1 = (0,01\cdot L)^2$ и $c_2 = (0,03\cdot L)^2$, величина $L$ описывает динамический диапазон значений пикселей.
\end{itemize}

Индекс SSIM варьируется от $-1$ до $1$, причем более высокие значения указывают на большее сходство между двумя изображениями.

Таким образом, критериями сравнения рассмотренных методов являются количественные величины метрик PSNR и SSIM, полученные путем сравнения оригинальных и восстановленных зашумленных изображений из открытых источников \cite{research}.

Результаты сравнения представлены в таблицах \ref{tab: comparison1}--\ref{tab: comparison2}.

\begin{table}[ht]
    \caption{Сравнение методов фильтрации изображений}
    \centering
    \begin{tabular}{|c|cc|cc|}
    \hline
    \multirow{2}{*}{Фильтр} & \multicolumn{2}{c|}{Изображение 1} & \multicolumn{2}{c|}{Изображение 2} \\ \cline{2-5} 
     & \multicolumn{1}{c|}{PSNR} & SSIM & \multicolumn{1}{c|}{PSNR} & \multicolumn{1}{c|}{SSIM} \\ \hline
    Среднего & \multicolumn{1}{c|}{26.75} & 0.564 & \multicolumn{1}{c|}{26.36} & 0.565 \\ \hline
    Гаусса & \multicolumn{1}{c|}{27.51} & 0.599 & \multicolumn{1}{c|}{27.41} & 0.629 \\ \hline
    Медианный & \multicolumn{1}{c|}{28.15} & 0.643 & \multicolumn{1}{c|}{27.70} & 0.646 \\ \hline

    \multicolumn{1}{|c|}{Билатеральный} & \multicolumn{1}{c|}{29.55} & \multicolumn{1}{c|}{0.782} & \multicolumn{1}{c|}{28.66} & 0.729 \\ \hline
    \end{tabular}
    \label{tab: comparison1}
\end{table}

\begin{table}[ht]
    \caption{Сравнение методов фильтрации изображений}
    \centering
    \begin{tabular}{|c|cc|cc|}
    \hline
    \multirow{2}{*}{Фильтр} & \multicolumn{2}{c|}{Изображение 3} & \multicolumn{2}{c|}{Изображение 4} \\ \cline{2-5} 
     & \multicolumn{1}{c|}{PSNR} & \multicolumn{1}{c|}{SSIM} & \multicolumn{1}{c|}{PSNR} & \multicolumn{1}{c|}{SSIM} \\ \hline
    Среднего & \multicolumn{1}{c|}{25.73} & 0.578 & \multicolumn{1}{c|}{23.22} & 0.542 \\ \hline
    Гаусса & \multicolumn{1}{c|}{26.84} & 0.637 & \multicolumn{1}{c|}{25.32} & 0.641 \\ \hline
    Медианный & \multicolumn{1}{c|}{26.58} & 0.642 & \multicolumn{1}{c|}{23.96} & 0.604 \\ \hline
    \multicolumn{1}{|c|}{Билатеральный}  & \multicolumn{1}{c|}{27.47} & 0.725 & \multicolumn{1}{c|}{24.51} & 0.686 \\ \hline
    \end{tabular}
    \label{tab: comparison2}
\end{table}

\section*{Выводы}
Исходя из приведенных выше данных сравнения методов устранения шумов на изображениях можно сделать следующие выводы:
\begin{itemize}
    \item среди линейных методов фильтр Гаусса показывает более высокую эффективность, чем метод среднего арифметического;
    \item среди нелинейных методов билатеральная фильтрация обеспечивает наивысшее качество восстановленного изображения;
    \item фильтр Гаусса и медианный фильтр обладают схожей эффективностью.
\end{itemize}